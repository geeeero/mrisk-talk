\documentclass{beamer}

% opacity bugfix: see http://tug.org/pipermail/pdftex/2007-December/007480.html
\pdfpageattr {/Group << /S /Transparency /I true /CS /DeviceRGB>>}

\usepackage[utf8]{inputenc}
\usepackage[T1]{fontenc}
\usepackage[ngerman]{babel}
\usepackage{tikz}
\usetikzlibrary{%
   arrows,%
   calc,%
   fit,%
   patterns,%
   plotmarks,%
   shapes.geometric,%
   shapes.misc,%
   shapes.symbols,%
   shapes.arrows,%
   shapes.callouts,%
   shapes.multipart,%
   shapes.gates.logic.US,%
   shapes.gates.logic.IEC,%
   er,%
   automata,%
   backgrounds,%
   chains,%
   topaths,%
   trees,%
   petri,%
   mindmap,%
   matrix,%
   calendar,%
   folding,%
   fadings,%
   through,%
   patterns,%
   positioning,%
   scopes,%
   decorations.fractals,%
   decorations.shapes,%
   decorations.text,%
   decorations.pathmorphing,%
   decorations.pathreplacing,%
   decorations.footprints,%
   decorations.markings,%
   shadows}
\usepackage{amssymb, amsmath, amsfonts, enumerate}
%\usepackage{bbold}
\newcommand\hmmax{0}
\usepackage{bm}
%\usepackage{dsfont}
\usepackage{pxfonts}
\usepackage{xcolor}
\usepackage{multirow}
\usepackage{rotating}
\usepackage{url}

\usepackage{hyperref}

\usetheme{Boadilla}
\usecolortheme{default}
%\usecolortheme{crane}

%\usetheme[secheader]{Boadilla}
\setbeamercovered{transparent}
\setbeamercovered{invisible}
\setbeamertemplate{navigation symbols}{}
%\setbeamertemplate{bibliography item}[text] % numbered references
\useoutertheme{infolines}
%\setbeamertemplate{headline}{}
\setbeamertemplate{footline}{\hspace*{5mm}\hfill\insertframenumber\hspace*{5mm}\vspace{3mm}}
\setbeamercolor{alerted text}{fg=orange!80!black}

\def\then{{\structure{$\rule[0.35ex]{2ex}{0.5ex}\!\!\!\blacktriangleright$}}}
\def\play{{\structure{$\blacktriangleright$}}}

\def\VaR{\text{VaR}}

\title{Markt(preis)risiko}
\author{Gero Walter\\ \url{http://www.geeeero.de}}
\institute{Vortrag bei der Informationsfabrik GmbH}
\date{6.~April 2017}

\begin{document}

\frame{
\titlepage
}

\begin{frame}{Markt(preis)risiko / market risk}
\begin{tikzpicture}[pfeil/.style={-latex', line width=1mm, color=orange!80!black, shorten <=1mm}]
\uncover<1>{%
\node at (0,0) {\parbox{\textwidth}{%
\begin{quote}
das Risiko, Verluste in Risikopositionen zu erleiden,\\
die durch Marktpreisbewegungen verursacht werden
\end{quote}
% the risk of losses in on- and off-balance sheet positions arising from movements in market prices. 
{\scriptsize\hfill Glossar der Bank für Internationalen Zahlungsausgleich, eigene Übersetzung}
}};}
\uncover<2>{%
\node (risikopos) at (1,2.5) {\parbox{0.6\textwidth}{\raggedleft\small was die Bank in ihren Büchern\\ (Handelsbuch, Anlagebuch) hat}};
\draw[pfeil] (risikopos) -- (0.4,0.85);
\node at (0,0) {\parbox{\textwidth}{%
\begin{quote}
das Risiko, Verluste in \alert{Risikopositionen} zu erleiden,\\
die durch Marktpreisbewegungen verursacht werden
\end{quote}
% the risk of losses in on- and off-balance sheet positions arising from movements in market prices. 
{\scriptsize\hfill Glossar der Bank für Internationalen Zahlungsausgleich, eigene Übersetzung}
}};}
\uncover<3>{%
\node (risikopos) at (1,2.5) {\parbox{0.6\textwidth}{\raggedleft\small was die Bank in ihren Büchern\\ (Handelsbuch, Anlagebuch) hat}};
\draw[pfeil] (risikopos) -- (0.4,0.85);
\node (marktpr) at (-2,-2) {\parbox{0.5\textwidth}{\small die Werte dieser Positionen\\ auf dem Markt ändern sich}};
\draw[pfeil] (marktpr) -- (-1.75,-0.25);
\node at (0,0) {\parbox{\textwidth}{%
\begin{quote}
das Risiko, Verluste in \alert{Risikopositionen} zu erleiden,\\
die durch \alert{Marktpreisbewegungen} verursacht werden
\end{quote}
% the risk of losses in on- and off-balance sheet positions arising from movements in market prices. 
{\scriptsize\hfill Glossar der Bank für Internationalen Zahlungsausgleich, eigene Übersetzung}
}};}
\end{tikzpicture}
%Risikopositionen: was die Bank in ihren Büchern (Handelsbuch, Anlagebuch) hat\\
%sMarktpreisbewegungen: die Werte dieser Positionen auf dem Markt ändern sich
\end{frame}

\begin{frame}{Risikoarten (nach MaRisk)}
\begin{tikzpicture}
[mybox/.style={rectangle,rounded corners,draw,fill=black!10,thick,inner sep=2pt,minimum size=6mm}, %font=\footnotesize},
 hv path/.style={thick, to path={-| (\tikztotarget)}},
 vh path/.style={thick, to path={|- (\tikztotarget)}}]
\node[mybox] (ges) at ( 0  , 0  ) {Gesamtrisiko};
\node[mybox] (mar) at (-4.5,-2  ) {Marktpreisrisiko};
\node[mybox] (cre) at (-1.5,-2  ) {Kreditrisiko};
\node[mybox] (liq) at ( 1.5,-2  ) {Liquiditätsr.};
\node[mybox] (ope) at ( 4.5,-2  ) {Operationelles R.};
\coordinate (uge) at (0,-1);
\path (ges) edge[thick] (uge)
      (uge) edge[hv path] (mar)
            edge[hv path] (cre)
            edge[hv path] (liq)
            edge[hv path] (ope);
\uncover<2->{%
\node (sub) at (-2,-4) {\parbox{0.3\textwidth}{Kursrisiko\\ Zinsänderungsrisiko\\ Währungsrisiko\\ Warenrisiko\\ \ldots}};
\path (mar) edge[vh path] (-3.9,-3.05)
            edge[vh path] (-3.9,-3.55)
            edge[vh path] (-3.9,-4.05)
            edge[vh path] (-3.9,-4.55)
            edge[vh path] (-3.9,-5.05);
}
\end{tikzpicture}
\end{frame}

%\begin{frame}{Systematisches vs.\ unsystematisches Risiko}
%\end{frame}

\begin{frame}{Messung des Marktpreisrisikos: \emph{Value at Risk} ($\VaR$)}

VaR wird genutzt zum (Banken-) internen Risikomanagement und zur Berechung von geforderten Eigenkapitalquoten\\ %von der Bankenaufsicht

$\VaR_{1-\alpha\%}(X)$: Wertverlust von $X$ über bestimmten Zeitraum ist in $\alpha\%$ der Fälle höchstens $\VaR_{1-\alpha\%}(X)$\\

$\VaR_{99\%}(X)$ = 99\%-Quantil der Verteilung des Verlusts (neg.\ Gewinn)\\

\play\ Brauche die Verteilung der Rendite! 
\end{frame}

\begin{frame}{Klassische Berechnungsmethoden}
\begin{itemize}
\item Varianz-Kovarianz-Methode (Delta-Normal-Ansatz): schätze sd aus Daten, NV-Annahme ergibt Quantil
\item Historische Simulation: Bestimme empirisches Quantil aus historischen Daten
\item Monte-Carlo-Simulation: Simuliere Renditen gemäß Verteilungsmodell, empirisches Quantil
\end{itemize}
\end{frame}

\begin{frame}{Fortgeschrittene Berechnungsmethoden}
\begin{itemize}
\item Filtered historic simulation
\item Vollparametrische Methoden
\item Extremwerttheorie-basierte Ansätze
\item Quantilsregression
\end{itemize}
\end{frame}

\begin{frame}{VaR ist kein kohärentes Risikomaß}

\end{frame}

\begin{frame}{}

\end{frame}

\end{document}
